%\documentclass[a4paper,11pt]{article}
\documentclass[a4paper,12pt,twoside,openright,titlepage]{article}

\usepackage[a4paper]{geometry}

\geometry{top=2.8cm, bottom=2cm, left=2.8cm, right=2.8cm, bindingoffset=1.5cm}

\usepackage[utf8]{inputenc}
\usepackage[spanish]{babel}  %Modify according your language


\decimalpoint

\makeatletter
\addto\shorthandsspanish{\let\esperiod\es@period@code}
\makeatother


\RequirePackage{verbatim}

 \usepackage{float}  % Para definir las tablas relativas a los tipos de requisitos

\usepackage{makeidx}         % allows index generation

\usepackage{graphicx}
\graphicspath{{Figures/}}


\usepackage{color}  
\usepackage[table]{xcolor}
\usepackage{multirow}

% ********************************************************************
% Re-usable information
%********************************************************************

%Debería de ser sustituido por la etiqueda dada al entregable
\newcommand {\tipoDoc} {Nombre del entregable}
%\newcommand {\tipoDoc} {Deliverable name}
\newcommand{\asignatura}{Ingeniería de los requisitos}  
%\newcommand{\asignatura}{Requirements Engineering}
\newcommand{\equipo}{Illuminatis 3.0}
%\newcommand{\equipo}{Team Id}
\newcommand{\primerAl}{Aitor García Luiz }
%\newcommand{\primerAl}{Name Surname}
\newcommand{\segunAl}{Juan Antonio Pérez Clemente }
\newcommand{\tercerAl}{Juan Antonio Rodríguez Baeza }




%\hyphenation{}



%%%% Definición de las cabeceras

\usepackage{fancyhdr}

\pagestyle{fancy}
\fancyhf{}

\fancyhead[LE]{ \textbf{\thepage} \ \ \ Equipo: \equipo   }
%\fancyhead[LE]{ \textbf{\thepage} \ \ \ Team: \equipo   }
\fancyhead[RO]{ Equipo: \equipo  \ \ \ \textbf{\thepage}}
%\fancyhead[RO]{ Team: \equipo  \ \ \ \textbf{\thepage}}
\fancyhead[LO]{\includegraphics[width=0.15\textwidth]{Cabecera/imagenes/inre.png} \miProyecto  }
\fancyhead[RE]{\miProyecto \includegraphics[width=0.15\textwidth]{Cabecera/imagenes/inre.png}   }

\renewcommand{\sectionmark}[1]{\markright{\textbf{\thesection. #1}}}

\setlength{\headheight}{2\headheight}


%%% Definiciones de nuevos  tipos
%\newcommand{\HRule}{\rule{\linewidth}{0.5mm}}
%\newcommand{\bigrule}{\titlerule[0.5mm]}


\newenvironment{textoazul}{\color{blue}}{\color{black}}   %Pendiente de revisar

\usepackage{tabulary}

\newcolumntype{P}[1]{>{\centering\arraybackslash}p{#1}} %Para tablas centradas dado ancho
\newcolumntype{M}[1]{>{\centering\arraybackslash}m{#1}}



\definecolor{gray97}{gray}{.97}
\definecolor{gray75}{gray}{.75}
\definecolor{gray45}{gray}{.45}
\definecolor{gray30}{gray}{.94}




%Para conseguir que en las páginas en blanco no ponga cabeceras
\makeatletter
\def\clearpage{%
  \ifvmode
    \ifnum \@dbltopnum =\m@ne
      \ifdim \pagetotal <\topskip
        \hbox{}
      \fi
    \fi
  \fi
  \newpage
  \thispagestyle{empty}
  \write\m@ne{}
  \vbox{}
  \penalty -\@Mi
}
\makeatother

\makeindex




\begin{document}

\begin{titlepage}
 

\newlength{\centeroffset}
\setlength{\centeroffset}{-0.5\oddsidemargin}
\addtolength{\centeroffset}{0.5\evensidemargin}
\thispagestyle{empty}

\noindent\hspace*{\centeroffset}\begin{minipage}{\textwidth}

\centering
\includegraphics[width=0.8\textwidth]{Cabecera/imagenes/logo_ual.jpg}\\[0.8cm]



{\Huge\bfseries \tipoDoc \\ }
\noindent\rule[-1ex]{\textwidth}{3pt}\\[3ex]

\textsc{ \Huge \asignatura \\[0.7cm]}

\end{minipage}

\vspace{2.3cm}
\noindent\hspace*{\centeroffset}\begin{minipage}{\textwidth}
\centering

\includegraphics[width=0.5\textwidth]{Cabecera/imagenes/logoesi.JPG}\\[0.1cm]

\textsc{---}\\
Almería, \today \\[1cm]

\textbf{Equipo:} {\equipo}\\[0.2cm]
%\textbf{Team:} {\equipo}\\[0.2cm]
\textbf{Miembros:}\\ {\primerAl\\ \segunAl \\ \tercerAl}\\[1cm]
%\textbf{Members:}\\ {\primerAl\\ \segunAl \\ \tercerAl}\\[1cm]


\end{minipage}

\end{titlepage}




\clearpage
\newcommand{\folder}{camino}  %para que el resto sean una redefinición

%Entregable simple --  Easy Assignment
%\renewcommand{\folder}{Simple}
%\newcommand{\fig}{Entregablegenerico/Figuras}

\section{Title 1}
\subsection{Title 1.1}
\subsubsection{title 1.1.1}

\begin{figure} [http]
\centering
\includegraphics [width=0.5\textwidth]{\fig/logo_esi.JPG}
\caption{Figura con el logotipo de la Escuela}\label{fig:reference}
\end{figure}


\begin{table}[htbp]
\begin{center}
    \begin{tabular}{|l|l}
        \hline
        \multicolumn{2}{|c|}{Nombre descriptivo}\\
%     \multicolumn{2}{|c|}{Identification name}\\
        \hline
        \hline
         \cellcolor{gray30}  Tipo de elemento &  Identificación del elemento a describir (Actor/proceso...)\\ 
%      \cellcolor{gray30}  Type & What is described?\\   
        \hline
         \cellcolor{gray30} Descripción	&  \\   
%      \cellcolor{gray30}  Description &   \\   
         \hline
         \cellcolor{gray30}  Comentarios	& comentarios adicionales \\   
%         \cellcolor{gray30} Comments	& additional comments \\
        \hline
  
\end{tabular}
\caption{nombre descriptivo} %Se puede eliminar o repetir el nombre descriptivo
%\caption{identification name} %this line can be omitted
\label{tabla:sencilla}
\end{center}
\end{table}

%Entregable con hoja de revisión - Assignment with review sheet

%\renewcommand{\folder}{Entregablegenerico}
%\newcommand{\fig}{Entregablegenerico/Figuras}

\section{Title 1}
\subsection{Title 1.1}
\subsubsection{title 1.1.1}

\begin{figure} [http]
\centering
\includegraphics [width=0.5\textwidth]{\fig/logo_esi.JPG}
\caption{Figura con el logotipo de la Escuela}\label{fig:reference}
\end{figure}


\begin{table}[htbp]
\begin{center}
    \begin{tabular}{|l|l}
        \hline
        \multicolumn{2}{|c|}{Nombre descriptivo}\\
%     \multicolumn{2}{|c|}{Identification name}\\
        \hline
        \hline
         \cellcolor{gray30}  Tipo de elemento &  Identificación del elemento a describir (Actor/proceso...)\\ 
%      \cellcolor{gray30}  Type & What is described?\\   
        \hline
         \cellcolor{gray30} Descripción	&  \\   
%      \cellcolor{gray30}  Description &   \\   
         \hline
         \cellcolor{gray30}  Comentarios	& comentarios adicionales \\   
%         \cellcolor{gray30} Comments	& additional comments \\
        \hline
  
\end{tabular}
\caption{nombre descriptivo} %Se puede eliminar o repetir el nombre descriptivo
%\caption{identification name} %this line can be omitted
\label{tabla:sencilla}
\end{center}
\end{table}

%Especificacion de los requisitos del software - Software requirement specification
\renewcommand{\folder}{SRS}
\newcommand{\fig}{Entregablegenerico/Figuras}

\section{Title 1}
\subsection{Title 1.1}
\subsubsection{title 1.1.1}

\begin{figure} [http]
\centering
\includegraphics [width=0.5\textwidth]{\fig/logo_esi.JPG}
\caption{Figura con el logotipo de la Escuela}\label{fig:reference}
\end{figure}


\begin{table}[htbp]
\begin{center}
    \begin{tabular}{|l|l}
        \hline
        \multicolumn{2}{|c|}{Nombre descriptivo}\\
%     \multicolumn{2}{|c|}{Identification name}\\
        \hline
        \hline
         \cellcolor{gray30}  Tipo de elemento &  Identificación del elemento a describir (Actor/proceso...)\\ 
%      \cellcolor{gray30}  Type & What is described?\\   
        \hline
         \cellcolor{gray30} Descripción	&  \\   
%      \cellcolor{gray30}  Description &   \\   
         \hline
         \cellcolor{gray30}  Comentarios	& comentarios adicionales \\   
%         \cellcolor{gray30} Comments	& additional comments \\
        \hline
  
\end{tabular}
\caption{nombre descriptivo} %Se puede eliminar o repetir el nombre descriptivo
%\caption{identification name} %this line can be omitted
\label{tabla:sencilla}
\end{center}
\end{table}

\end{document}
